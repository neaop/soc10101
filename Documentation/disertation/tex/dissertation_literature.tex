\section{Literature Review}
	\subsection{Introduction}
		As stated in the introduction, at the time of writing there exists no published papers, journals, or studies concerning the detection or assessment of Dyslexia via the use of Fitts’s Law. While literature relating directly to this project may be non-existent, there is a substantial amount of existing material concerning both Fitts’s Law and Dyslexia detection, as well as the detection of Learning and Developmental Difficulties via applications inspired by Fitts’s Law. 
		
	\subsection{Fitts’s Law}
		In 1954, Paul M. Fitts published his thesis, consisting of a study of human motor skills and how the average person’s motor speed and accuracy varied proportionally to a number of test variables and could be accurately predicted \cite{Fitts1954}. Fitts originally conducted his study by having test participants carry out a number of physical tests, such as tapping a stylus between two conductive strips, moving a set of washers from one pin to another, and moving a number of pegs from one set of holes to another \cite{Fitts1954}. While Fitts’s original study may appear  dated in style and delivery, the hypothesis and results of the experiments which were carried out proved that the prediction of an individual's ability to execute motor tasks was not only feasible but extremely accurate \cite{Fitts1954}. Fitts’s research gained so much traction that it gained partial support from the United States Air Force, an organisation where the assessment of a person’s motor skills is of critical importance \cite{Fitts1954}.
		
		Fitts’s original research was conducted before the commercial availability of personal computers, with the intent of predicting how quickly and accurately an individual could carry out a physical task , but his work and analysis can also be easily and readily applied to digital tasks. A number of reassessments of Fitts’s original work have been conducted since its publishing, with the aims of refining and readjusting the original equations for modern usages.
		
		The first investigation of the use of Fitts’s Law in a HCI context was conducted in 1978 \cite{Card1978}. The study in question was a comparison of the usability of a number of computer input methods, exploring the speed that users could move an onscreen cursor to various targets and the amount of errors made in the process. The devices used in the study were a mouse, joystick, and keyboard. The results of the investigation were as expected, with the mouse being the usable input device in terms of speed accuracy, but more importantly - for this project - it demonstrated that Fitts’s Law could be applied to computer use, even when physical movement of a device was taking place in the form of using keyboard keys.
		
		One of the more comprehensive reviews of Fitts’s Law is a study provided by Ian MacKenzie \citeyear{MacKenzie1995}. MacKenzie has produced a number of studies in relation to HCI, a field in which modeling human motor skills is of extreme importance. MacKenzie raises the point that the tests conducted in Fitts’s original study are entirely one-dimensional, i.e. the distance between the two targets as well as their widths were always altered along the same axes. This means that the ID can change variably depending on the angle of approach towards non-uniform targets, such as rectangular buttons or words. Fortunately, the Dot-to-Dot Task targets are all uniform circles, meaning that their width always remains the same, regardless of the angle of approach. MacKenzie also provides details into a slightly altered MT prediction equation in the form of,
		
		\[MT = a+b \cdot ID = a+b \cdot \log_2 \left( \frac{2D}{W} \right) \]
		
		Values \(a\) and \(b\) are calculated via regression analysis, where the results from previous tests are plotted in order to produce a more accurate prediction model.
		
		Fitts’s Law has been used in a number of diverse projects, even including the evaluation of motor skills in monkeys \cite{Ifft2011}. Two monkeys were trained to move a cursor to various sized targets using a joystick and the MT required was recorded. It was found that the monkey’s IP was within the expected Fitts’s prediction, providing further proof to Fitts’s Law’s accuracy, even across species.
		
		It should be noted that in the studies presented previously, the ID of the tests was scaled in a uniform manner, thus the IDs of the tests would be 1, 2, 3, etc. The Dot-to-Dot Task, on the other hand, does not consist of sectors of gradually scaling ID, which affects the viability of the use of regression analysis and ultimately the altered equation for this project.
		
		
	\subsection{Detection via Fitts’s Law}
		As the previous literature shows, Fitts’s Law has shown its extended used as a metric for the measurement of human motor abilities, both for physical and digital-based tasks. A number of applications based on Fitts’s Law have been utilised in the past for the detection of a number of Learning  and Developmental Difficulties.
		
		A study conducted in the Netherlands \cite{Smits-Engelsman2003}, which investigated the performance of Dyspraxic children made heavy use of Fitts’s Law, from the design of tests conducted, to the assessment of the individuals. Dyspraxia - also known as DCD - is a neurological disorder that primarily affects an individual’s movements and coordination. The investigation tested a number of children by having them draw straight lines of various lengths using a drawing tablet and stylus, not dissimilar to the Dot-to-Dot Task used in this project. The study in question found that the individual's IP varied very little, regardless of whether the children were Dyspraxic or not. It was found that there was a significant error difference between the two groups of children - while Fitts’s Law may be an appropriate metric for measurement, it may not always be appropriate for the differentiation between children with and without certain Learning Difficulties. This is of particular note, as Dyspraxia has the potential to affect fine motor skills to potentially severe degrees, implying that the IP between the two groups would be distinct enough for clear detection.
		
		
		Fitts’s Law has also been used to assess individuals with physically induced motor impairments, for example, due to injury or partial paralysis \cite{Maruff2000}. The study had two groups of participants draw vertical lines of different lengths, one group consisting of individuals suffering from arm injuries or paralysis and the other consisting of healthy individuals feigning motor impairment. It was found that the individuals suffering from injuries produced an IP conforming to Fitts’s Law, whereas the group feigning injury were below the anticipated IP. While this may indicate that Fitts’s Law’s accuracy is refined, it becomes accurate when an individual is attempting to falsify results, but it should be noted that there was a major disparity in terms of the groups sizes, with only one injured subject and ten healthy (albeit feigning impairment) subjects.
		
		This was a second study conducted to compare the motor abilities of Dyspraxic and non-Dyspraxic children \cite{Wilson2001}. The study consisted of drawing a straight line to a target and then back to the initial starting location. A number of different target sizes were used, allowing for number of different difficulties to be tested and regression analysis to be performed on the results. Two kinds of experiment were performed with the participants: a standard motor skill test, in which the children physically drew a line, and an imagined test where the children would envision themselves drawing the line and state when they had completed the imaginary movement. It was found once again that the Dyspraxic childrens’ individual IPs were within the expected limits of Fitts’s Law as were the non-Dyspraxic childrens’ IPs. Interestingly, even though their physical movements were within the expected tolerance, the Dyspraxic childrens’ imagined movements did not align with the Fitts’s Law expectation.
		
		While the study itself had little to do with the assessment of Fitts’s Law or its use as an assessment tool, it does show that it is considered a credible form of detection for some Learning Difficulties.
		
\newpage