\begin{abstract}
	This dissertation records and describes the process and evaluation undertaken, while attempting to discern whether Fitts's Law analysis of Movement Times is a valid metric for the identification of  Dyslexia. Described within is motivation for conducting the study, the design and methodology of the technologies and systems used to collect and manipulate data, descriptions of how data gathered during the project was analysed, and the results and conclusions that can be drawn from this analysis.
	
	The aim of the project was to provide information to aid in the assessment of Dyslexia in children from a younger age, allowing for appropriate support to be provided during key developmental stages of their lives. A substantial amount of movement data was collected from volunteers, which was later processed and evaluated using Fitts's Law, as well as regression analysis and statistical analysis. The results of said testing indicate that the Fitts's Law Index of Performance varies very little between Dyslexic and non-Dyslexic task participants, so an alternative metric for detection in the form of analysis of the number of errors made during the motor test is suggested.
\end{abstract}

\pagebreak