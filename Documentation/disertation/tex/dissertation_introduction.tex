\section{Introduction}
	\subsection{Background and Context}
		\subsubsection{Dyslexia}
			\texttt{Brief description of Dyslexia and the methods used for it identification}\\
			Dyslexia is present in around 10\% of the British population to some degree; about 4\% of the population have severe Dyslexia \cite{BritishDyslexiaAssociation} \cite{NHSChoices}. While the understanding and acceptance of Dyslexia has improved greatly in recent years, the assessment of whether a person has Dyslexia has remained relatively stagnant. 
			Dyslexia is one of the few learning difficulties which does not have an assessment funded by the NHS. The current and most accurate form of assessment remains in the form of traditional “pen-and-paper” tests. \cite{BritishDyslexiaAssociation2016} \cite{NHSChoicesa}
		
		\subsubsection{Dot-to-Dot Task}
			\texttt{Describe the Dot-to-Dot Task. Bring attention to Jon's observation of dyslexic children having different outcomes}\\
			The Dot-to-Dot Task is a system developed by Professor Jon Kerridge [ref]. The goal of the application is to measure a child's ability to draw lines between a series of point as they appear on the screen. Kerridge hypothesises that a child with dyslexia will have measurably different results from a child that does not have dyslexia.
		
		\subsubsection{Fitts's Law}
			\texttt{Description of Fitts's Law. Although mainly used by designers for ergonomics, the Dot-to-Dot test can easily be analysed with Fitts's.}\\
			Fitts's Law is a form of analysis used to predict the "difficulty of movement" when using a digital input. Paul Fitts developed the theory in 1954, and it has since been utilised globally to aid in  the design of ergonomic and easy-to-use computer interfaces.\\ 
			As the Dot-to-Dot Test comprises of drawing lines between points - Fitts's analysis can be applied to the results.
		
		\subsubsection{Aims and Goals}
			\texttt{Outline the aims and goals of this project - deliverables}\\
			The goal of this project is to determine whether or not Fitts's Analysis is valid method of dyslexia detection.\\
			There are a number of precursors to determining whether Fitts's is a suitable discriminant, including;
			\begin{itemize}
				\item Collecting Dot-to-Dot data from Dyslexic and Non-Dyslexic participants for use as a control group
				\item Fitts Analysis of Dyslexic and Non-Dyslexic data for significant differences
				\item Evaluation of Fitts's analysis via a third, experimental group
			\end{itemize}
		
\newpage