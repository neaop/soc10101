\section{Conclusion and Future Work}
	The aim of this project was to investigate whether or not Fitts’s Law was a suitable method for the detection of Dyslexia. Within the confines of the experiments that have been run and analysed in this project, it appears not to be a valid Dyslexia-detection discriminant method.
	
	The premise of this project was based upon the anecdotal observation of primary school children participating in the Dot-to-Dot Task. There appeared to be a correlation between reading ability and Dot-to-Dot Task performance of the children, that could unfortunately not be officially analysed, due to the age of the children, as well as a lack of hard evidence. This project’s goal was to recreate and investigate in a controlled and processed manner the theory that the Dot-to-Dot Task could be used to detect Dyslexia. However, this was found not to be the case.
	While it may be entirely possible to detect Dyslexia via Fitts’s Law there are a number of nuances and issues that did not translate from the original hypothesis to this project. These are as follows.
	
	\subsection{Fitts’s Law}
		Fitts’s Law was the metric of performance chosen for investigation, as it has been shown as a reliable tool for the prediction of human motion, as well as measuring performance. While it is true that Fitts’s Law can easily determine the difficulty and execution of a movement along a straight line and the fact that the Dot-to-Dot Task patterns consist of straight lines, the task was clearly not designed with Fitts’s Law in mind and, as such, Fitts’s Law may not have been the optimal method to measure performance of the Dot-to-Dot Task. 
		
		As is shown in the Literature Review, a typical Fitts’s Law experiment consisted of having participants drawing a single horizontal or vertical line between two targets - the size and distance between each target being altered in subsequent tests so data could be collected for a number of IDs. The Dot-to-Dot Task consists of participants drawing a continuous sequence of lines, at various angles, between identically-sized targets that are spaced at relatively similar distances. It became clear that the Dot-to-Dot Task was not a typical Fitts’s experiment - as Fitts’s Law does not take into account the angle of movement or continuous drawing of lines.
	
	\subsection{Age of Participants}
		The second major difference between the observed childrens’ performance and the experiment performed in this project is the age of the participants. The original observation of the Dot-to-Dot Task and reading skill performance consisted of children from ages 4 to 11, while the ages of the participants in this project ranged from 19 to 68. While it is true that Dyslexia cannot be cured, as such, the severity and manifestations in which it affects an individual can be mitigated and reduced over time, with patience and learning techniques. It is entirely possible that the Dyslexic individuals that participated in this study had a sufficient understanding and experience of Dyslexia and thus their performance in the Dot-to-Dot task was not affected as noticeably as the children who have also participated in the task. The children that participated in the original study were too young to have an official Dyslexia assessment, meaning it is entirely possible that many of the children with low reading, writing and Dot-to-Dot Task performance were not Dyslexic. At the early ages that these children were, their levels of cognitive development can vary drastically among peers. 
		
		In summary, Fitts’s Law does not appear to be a credible form of Dyslexia assessment when it is used to evaluate the results of the Dot-to-Dot Task in its current state. This is in part due to Fitts’s Law as well as the current state of the Dot-to-Dot task. Fitts’s Law is an extremely accurate human-movement prediction tool, but as was discussed in the Literature Review, individuals with cognitive and developmental issues that affect their motor skills typical abide to the predictions suggested by Fitts’s Law. The second issue was the fact the Dot-to-Dot Task only had two patterns available to test participants, meaning there was not a broad enough spectrum of difficulties to perform adequate regression analysis.
		
	\subsection{Future Work}
		There exists a number of ways in which this project could be continued into the future, either focusing on further research into Fitts’s Law or attempting a different type of analysis, building upon the current Dot-to-Dot data.
		
		If further experimentation into the use of Fitts’s Law is to take place, the Dot-to-Dot Task would require a significant amount of refactoring and reformatting to allow for more accurate and credible Fitts’s analysis. While the back-end of the application remains an excellent data collection tool, the patterns that the participants trace would require an almost complete redesign. Current patterns all use a fixed target width of twenty pixels - varying the size of the targets allows for a variation in the ID of a movement, without the need for excessive distances between the targets. 
		
		A second issue for consideration is the continuous nature of the Dot-to-Dot patterns. Typically, Fitts’s experiments have participants complete a single move at time, whereas the Dot-to-Dot patterns consist of seven to eight different movements in a single pattern. If the patterns were redesigned to be individual movements, the Dot-to-Dot Task would be similar to previously conducted Fitts’s experiments, allowing for a more concise Fitts’s analysis.
		
		Another area of investigation that could be conducted in the future would be research into how the angle of sector affects its ID. The analysis of the Dot-to-Dot results revealed a significant difference in the number of  errors made by Dyslexic experiment participants in upward sloping sectors that directly followed downward sloping sectors. Patterns could be designed to further test this theory, with the aim of evaluating Fitts’s Law’s usability in multiple movement scenarios.
		
		Should future research focus on Dyslexia identification via the Dot-to-Dot Task in it’s current state - i.e. without altering its design and collecting additional data - a number of different identification techniques could be employed. One such method of identification could be the use of an Artificial Neural Network (ANN): a form of  deep learning artificial intelligence. A neural network can be trained on the previously collected Dot-to-Dot Task results of the Dyslexic and non-Dyslexic participants and learn to identify unique traits or features in each data set. Neural networks have been proven to be extremely prudent at spotting similar traits between different data sets. However, care must be taken to ensure that the system does not detect a false positive in a data set, leading to false and invalid assessments.
		
		While Fitts’s Law has been proven as an effective and reliable model of human movement, it is likely that other alternatives exist, some of which may be more effective than Fitts’s Law.
		
		\newpage
		
	\section{Learning Outcomes}
		This project had a number of learning outcomes attached to it, the completion of which provide evidence for a well-planned and executed project. The learning outcomes and evidence of their completion are presented thusly.
		
		\textbf{L.O.1: Manage a substantial individual project, including planning, documentation and control.}
		To ensure the project remained on track, weekly meetings with the project supervisor were arranged and attended. Evidence of this was recorded in a project blog that was updated weekly, with the progress made since the last supervisor meeting. The minutes of the supervisor meetings were appended to each post as a comment.
		
		At the beginning of the project, a basic plan was drafted in the form of a Initial Project Overview (appendix). This document outlined the deliverables of the project, as well as the goals to be achieved as the project progressed. 
		
		Further project management and advice was provided in the form of a mid-way meeting with the project supervisor and second marker.
		
		\textbf{L.O.2: Construct a focussed problem statement and conduct a suitable investigation, including literature or technology review, into the context of that problem.}
		As the premise of the project was a novel Dyslexia assessment method, finding relevant examples measuring the efficiency and effectiveness of implementation was extremely difficult. A major concern was the fact that at the time of writing, there existed no available literature in regards to detection of Dyslexia via Fitts’s Law - existing literature typically being the cornerstone of any project of this scale.
		
		\textbf{L.O.3: Demonstrate professional competence by applying appropriate theory and practice to the analysis, design, implementation and evaluation of a non-trivial set of deliverables.}
		The prime goal of this dissertation was to investigate Fitts’s Law’s usability as an analysis tool for the detection of Dyslexia. Evidence of Fitts’s Law’s inability to differentiate Dyslexic and non-Dyslexic task participants has been provided, as well as alternative methods of the detection that could be attempted in the future, hopefully with more success. While the term ‘investigate’ may imply a vague or undefined project, the scope of the investigation was very well defined. The scope was also specified by the available and collected data in the form of the Dot-to-Dot Task, rather than using movement data from a number of different tasks and experiments. 
		
		\textbf{L.O.4: Show a capacity for self-appraisal by analysing the strengths and weaknesses of the project process and outcomes with reference to the initial objectives and to the work of others.}
		In terms of prioritising and adapting to the needs of the project, the flow and direction of the project had to be readjusted several times throughout its progress. When it was found that the regression analysis results were skewed due to the lack of diversity in the sector difficulties, an alternative solution was found in the form of Fitts’s original equation, published in his initial thesis on the matter.
		Focus was also changed when it was found that the IP varied very little between Dyslexic and non-Dyslexic task participants, with an alternative metric of evaluation being suggested in the form of percentage of errors.
		
\newpage